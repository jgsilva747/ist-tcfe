\section{Simulation analysis}
\label{sec:simulation}

\subsection{Simulating the AC/DC converter for 10 periods}
As said in the introduction, the first step to this laboratory assignment was to simulate a simple AC/DC converter in NGSpice. The circuit features an ideal transformer, using a current controlled voltage source as well as a voltage controlled current source as explained by the professor in a previous lecture, as well as an envelope detector and a voltage regulator. \par
However, because it is known that the ideal transformer creates a new voltage and current with the same frequency as the original, instead of controlled sources, a simple sinusoidal voltage source would sufice.
This AC/DC converter was simulated for 10 periods and all the analysis were made measuring on a $5e-5$ step in order to evaluate at least 1000 points during the 10 periods. In order to calculate this step, the frequency of the AC sourcecwas used to know the period and then multiplied by 10 in order to get the total time. We then divided the total time by 1000 points and made the step even smaller than that in order to make sure it had more than 1000 points but not too small that the program ran slowly. \par


\subsection{Output voltage level}
After describing the circuit we made NGSpice measure the average output voltage and using a transient analysis we plotted both the average and the signal of the output voltage in the same graph. \par
The table and the graph below show the obtained results.

\begin{table}[H]
  \centering
  \begin{tabular}{|l|r|}
    \hline    
    {\bf Name} & {\bf Value [V]} \\ \hline
    \input{average_tab}
  \end{tabular}
  \label{tab:average}
\end{table}

\begin{figure}[H] \centering
\includegraphics[width=0.7\linewidth]{../sim/transient1.pdf}
\caption{Plot of the average and the signal of the Output Voltage.}
\label{fig:transient1}
\end{figure}

\subsection{Output of the Envelope Detector and voltage Regulator circuits}
The output voltages of both the Envelope Detector as well as the Voltage Regulator circuits were plotted and put each in a different graph as well as a graph with both voltages plotted. \par
The three graphs are in the images below.

\begin{figure}[H] \centering
\includegraphics[width=0.7\linewidth]{../sim/transient2.pdf}
\caption{Envelope Detector Output Voltage.}
\label{fig:transient2}
\end{figure}

\begin{figure}[H] \centering
\includegraphics[width=0.7\linewidth]{../sim/transient3.pdf}
\caption{Voltage Regulator Output Voltage.}
\label{fig:transient3}
\end{figure}

\begin{figure}[H] \centering
\includegraphics[width=0.7\linewidth]{../sim/transient4.pdf}
\caption{Envelope Detector and Voltage Regulator Output Voltages.}
\label{fig:transient4}
\end{figure}

\subsection{Output voltage ripple}
We then made NGSpice measure the output voltage ripple, that is the difference between the maximum and the minimum values of the signal. \par
The result are the following:

\begin{table}[H]
  \centering
  \begin{tabular}{|l|r|}
    \hline    
    {\bf Name} & {\bf Value [V]} \\ \hline
    \input{ripple_tab}
  \end{tabular}
  \label{tab:ripple}
\end{table}

\subsection{$v_0 - 12$ plot}
Lastly, we plotted $v_0 - 12$, which corresponds to the output AC component plus the DC deviation, and calculated the deviation, using the mean value. \par
The plot can be seen in the image below, as well as the deviation in the table that follows.\par

\begin{figure}[H] \centering
\includegraphics[width=0.7\linewidth]{../sim/transient5.pdf}
\caption{Output AC component + DC deviation.}
\label{fig:transient5}
\end{figure}

\begin{table}[H]
  \centering
  \begin{tabular}{|l|r|}
    \hline    
    {\bf Name} & {\bf Value [V]} \\ \hline
    \input{meanv012_tab}
  \end{tabular}
  \label{tab:meanv012}
\end{table}