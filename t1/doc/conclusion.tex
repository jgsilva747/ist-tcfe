\section{Conclusion}
\label{sec:conclusion}

In this laboratory assignment the objective of analysing this circuit has been
achieved. The analysis of the linearly dependent sources (one current controlled voltage source and one voltage controlled current source) have been performed both theoretically using the Octave maths tool and by circuit simulation using the
Ngspice tool. The simulation results matched the theoretical results
precisely. The reason for this perfect match is the fact that this is a
straightforward circuit containing only linear components, so the theoretical
and simulation models cannot differ. For more complex components, the
theoretical and simulation models could differ but this is not the case in this
work. This fact can easily be justified due to the nature of this simulation, since Ngspice uses the exact same methods as those that were used in the theoretical calculations to solve the circuit, therefore it was expected to output the same results as those calculated before. 
 \par Moreover, since Ngspice provides a simulation, all the resistors are perfect, and so are the branches, nodes and 'wires', with no electrical resistance  and no energy wasted on heating. Furthermore, since there are no real measurements, it is found that no errors emerge from inaccurate readings, and hence the theoretical results being precisely the same as the simulated ones.


\section{Theoretical Note}
\label{sec:final_note}

\begin{figure}[h] \centering
\includegraphics[width=0.7\linewidth]{meme.JPG}
\label{fig:Meme}
\end{figure}